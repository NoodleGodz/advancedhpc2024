\documentclass{article}


\title{Labwork 1: Gradient Descent}

\begin{document}

\maketitle

\setlength\parindent{0pt}

\section{Tasks:}

Apply Gaussain blur to this image:

\begin{figure}[H]
    \centering
    \includegraphics[width=1\linewidth]{../../images/image3.jpg}
\end{figure} 
Information: 
\begin{itemize}
    \item Height: 180
    \item Weight: 320
    \item Channel: 3
\end{itemize}

\section{Implementation}
First, we define our kernel with a function:

\begin{lstlisting}[language=Python]
    def gauss_kernal():
    sigma = 2
    sumv = 0
    kernal = []
    for i in range(-3 , 4 ):
        row = []
        for j in range(-3 , 4 ):
            g = (1 / (2 * math.pi * sigma ** 2)) * math.exp(-(i**2 + j**2) / (2 * sigma ** 2))
            row.append(g)
            sumv += g
        kernal.append(row)

    for i in range(7):
        for j in range(7):
            kernal[i][j] /= sumv
    return kernal
\end{lstlisting} 


Hard coded for the 7x7 kernel.

Then i do the convolution sum of that with each pixel using offset (-3,3) with our kernal
\begin{lstlisting}[language=Python]
    for h in range(x_step):
    for w in range(y_step):
            for c in range(C):  
                w_sum = 0
                for i in range(7):
                    for j in range(7):
                        w_sum += hostInp[h + i][w + j][c] * k[i][j]
                output[h][w][c] = int(w_sum)
\end{lstlisting} 
 
Time elapsed in CPU : 8.981184720993042

\section{GPU}




In this labwork, we did this, that, and the result is this.

An interesting finding is that.....

\end{document}
